
\documentclass{article}
\usepackage{amsmath}
\usepackage[margin=2cm]{geometry}
\usepackage[T1]{fontenc}
\usepackage{listings}

\title{3D Random Walk}
\author{LYU Liuke}
\date{PHYS3061: Lab Report 1}

\begin{document}

\maketitle
\tableofcontents
\clearpage

\section{Introduction}

\subsection{Random Number Generator}

The random number generator implemented here is a simple Linear Congruential
Generator(LCG), which is generally a recursive mapping within a certain integer
domain. for a set of parameters (a,c,m):

\begin{align*}
  x_{N+1} &= (x_N * a + c) \mod m \\
  \xi \colon \{1,2,3,\dots,m-1\} &\to \{1,2,3,\dots,m-1\}
\end{align*}

Some special sets of (a,c,m) give seemingly very random and uniform distribution
of results in recursive mapping. Based on this feature, we consider this recursive 
method as a random number generator. \\

In this experiment (a=1559, c=647, m=13229) are chosen. 
The main characteristic of these parameters is that they are all primes, in order
to try to avoid periodicity behaviors in the random sequance.
This special set is chosen by testing different combinations of (a,c,m) in a
small parameter space.

\subsection{Random Walk}
  
A random walk, in simple terms, is a random sequance representing the 
accumulative sum of a random variable. In a simple 3D case:

\begin{align*}
  \vec{X_N} &= \sum_{i=1}^{n} \vec{s_i} \\
  \mbox{where} \quad  \vec{s} \in V &= \{(1,0,0), (-1,0,0), (0,1,0), (0,-1,0), (0,0,1),(0,0,-1)\} \\
  \mbox{Assigning} \quad P(\vec{v}) &\equiv 1/6 \quad \mbox{for}  \quad \vec{v} \in V
\end{align*}

Probability analysis give this famous relationship:

\begin{align*}
  <|\vec{X_N}|^2> = \sqrt{N} 
\end{align*}

This relationship can be verified by a Monte Carlo Method, averaging the results
over a large number of samples. Alternatively, we can use it to evaluate the 
randomness of our LCG generator.
\section{Code}

\subsection{Linear Congruential Generator}

\lstinputlisting[language=Python,firstline=8, lastline=23]{lab1.py}

LCG is the linear congruential generator with parameters a=1559, c=647,
m=13229. The seed of the generator is determined by decimal time. The structure of it
is a python generator, which is a sequance of predefined operations. This gives a good 
structure to continuously generate any length of random numbers and also avoids defining
a global state variable. The outputs of this function (after "yield") is normalized by
dividing it by m. \\

\subsection{3D Random Walk}

\lstinputlisting[language=Python,firstline=28, lastline=38]{lab1.py}

For a 3 dimensional random walk, every steps needs to be a vector whose direction is
a random choice among x, y, z axes, and value a choice between (-1,1). That's why we 
need to implement a random choice function to convert a random number between (0,1) to 
a random choice in a list.



\lstinputlisting[language=Python,firstline=40, lastline=54]{lab1.py}

\section{Application}

\subsection{P\'olya's Random Walk Problem}

\section{Results and Analysis}

\section{Conclusion}

\end{document}

\iffalse
  Your Report should include
- Brief introduction of random number generator and random walk (less than one page)
- Present the parameter you used and outline your codes
- Results and analysis. (Proper figures or tables should be included to show the results)
- Conclusion

\fi
