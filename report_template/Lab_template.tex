%        File: report.tex
%     Created: Mon Nov 27 07:00 PM 2017 H
% Last Change: Mon Nov 27 07:00 PM 2017 H

% defining variables to be used
%
\documentclass{article}
\usepackage{amsmath}
\usepackage[margin=2cm]{geometry}
\usepackage[T1]{fontenc}

\title{3D Random Walk}
\author{LYU Liuke}
\date{PHYS3061: Lab Report 1}

\begin{document}

\maketitle
\tableofcontents

\section{Introduction}

\subsection{Random Number Generator}

The random number generator implemented here is a simple Linear Congruential
Generator(LCG), which is generally a recursive mapping within a certain integer
domain. for a set of parameters (a,c,m):

\begin{align*}
  x_{N+1} &= (x_N * a + c) \mod m \\
  \xi \colon \{1,2,3,\dots,m-1\} &\to \{1,2,3,\dots,m-1\}
\end{align*}

\subsection{Random Walk}

\section{Code}

\subsection{Linear Congruential Generator}

\subsection{3D Random Walk}

\subsection{P\'olya's Random Walk Constants}

\section{Results and Analysis}

\section{Conclusion}

\end{document}

\iffalse
  Your Report should include
- Brief introduction of random number generator and random walk (less than one page)
- Present the parameter you used and outline your codes
- Results and analysis. (Proper figures or tables should be included to show the results)
- Conclusion

\fi
